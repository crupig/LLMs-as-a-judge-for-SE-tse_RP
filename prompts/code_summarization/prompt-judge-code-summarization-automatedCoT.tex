STEP 1:
You will be provided with a <LANGUAGE> function ("Function") and a textual summary of it ("Comment"). The goal of the Comment is to document the functionality implemented in the Function.

# Function
<FUNCTION>

# Comment
<COMMENT>

# Question: evaluate the Comment across three Evaluation Criteria

* Content adequacy: the extent to which the comment summarizes all information that can be inferred from the source code.
* Conciseness: the extent to which the comment contains unnecessary information.
* Fluency & Understandability: the extent to which the comment is easy to read and understand.

# Reasoning: Let's think step by step.

STEP 2:
You will be provided with a <LANGUAGE> function ("Function") and a textual summary of it ("Comment"). The goal of the Comment is to document the functionality implemented in the Function.

# Function
<FUNCTION>

# Comment
<COMMENT>

# Question: evaluate the Comment across three Evaluation Criteria

* Content adequacy: the extent to which the comment summarizes all information that can be inferred from the source code.
* Conciseness: the extent to which the comment contains unnecessary information.
* Fluency & Understandability: the extent to which the comment is easy to read and understand.

# Reasoning: Let's think step by step.
<ANALYSIS>

# Now rate all the three criteria based on the reasoning. For each criterion, provide a score on a scale from 1 to 5, and no other text:

1. Very poor
2. Poor
3. Fair
4. Good
5. Very good

# Rating: